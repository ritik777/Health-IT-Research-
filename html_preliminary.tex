% Options for packages loaded elsewhere
\PassOptionsToPackage{unicode}{hyperref}
\PassOptionsToPackage{hyphens}{url}
%
\documentclass[
]{article}
\usepackage{lmodern}
\usepackage{amssymb,amsmath}
\usepackage{ifxetex,ifluatex}
\ifnum 0\ifxetex 1\fi\ifluatex 1\fi=0 % if pdftex
  \usepackage[T1]{fontenc}
  \usepackage[utf8]{inputenc}
  \usepackage{textcomp} % provide euro and other symbols
\else % if luatex or xetex
  \usepackage{unicode-math}
  \defaultfontfeatures{Scale=MatchLowercase}
  \defaultfontfeatures[\rmfamily]{Ligatures=TeX,Scale=1}
\fi
% Use upquote if available, for straight quotes in verbatim environments
\IfFileExists{upquote.sty}{\usepackage{upquote}}{}
\IfFileExists{microtype.sty}{% use microtype if available
  \usepackage[]{microtype}
  \UseMicrotypeSet[protrusion]{basicmath} % disable protrusion for tt fonts
}{}
\makeatletter
\@ifundefined{KOMAClassName}{% if non-KOMA class
  \IfFileExists{parskip.sty}{%
    \usepackage{parskip}
  }{% else
    \setlength{\parindent}{0pt}
    \setlength{\parskip}{6pt plus 2pt minus 1pt}}
}{% if KOMA class
  \KOMAoptions{parskip=half}}
\makeatother
\usepackage{xcolor}
\IfFileExists{xurl.sty}{\usepackage{xurl}}{} % add URL line breaks if available
\IfFileExists{bookmark.sty}{\usepackage{bookmark}}{\usepackage{hyperref}}
\hypersetup{
  pdftitle={Preliminary\_analysis},
  hidelinks,
  pdfcreator={LaTeX via pandoc}}
\urlstyle{same} % disable monospaced font for URLs
\usepackage[margin=1in]{geometry}
\usepackage{color}
\usepackage{fancyvrb}
\newcommand{\VerbBar}{|}
\newcommand{\VERB}{\Verb[commandchars=\\\{\}]}
\DefineVerbatimEnvironment{Highlighting}{Verbatim}{commandchars=\\\{\}}
% Add ',fontsize=\small' for more characters per line
\usepackage{framed}
\definecolor{shadecolor}{RGB}{248,248,248}
\newenvironment{Shaded}{\begin{snugshade}}{\end{snugshade}}
\newcommand{\AlertTok}[1]{\textcolor[rgb]{0.94,0.16,0.16}{#1}}
\newcommand{\AnnotationTok}[1]{\textcolor[rgb]{0.56,0.35,0.01}{\textbf{\textit{#1}}}}
\newcommand{\AttributeTok}[1]{\textcolor[rgb]{0.77,0.63,0.00}{#1}}
\newcommand{\BaseNTok}[1]{\textcolor[rgb]{0.00,0.00,0.81}{#1}}
\newcommand{\BuiltInTok}[1]{#1}
\newcommand{\CharTok}[1]{\textcolor[rgb]{0.31,0.60,0.02}{#1}}
\newcommand{\CommentTok}[1]{\textcolor[rgb]{0.56,0.35,0.01}{\textit{#1}}}
\newcommand{\CommentVarTok}[1]{\textcolor[rgb]{0.56,0.35,0.01}{\textbf{\textit{#1}}}}
\newcommand{\ConstantTok}[1]{\textcolor[rgb]{0.00,0.00,0.00}{#1}}
\newcommand{\ControlFlowTok}[1]{\textcolor[rgb]{0.13,0.29,0.53}{\textbf{#1}}}
\newcommand{\DataTypeTok}[1]{\textcolor[rgb]{0.13,0.29,0.53}{#1}}
\newcommand{\DecValTok}[1]{\textcolor[rgb]{0.00,0.00,0.81}{#1}}
\newcommand{\DocumentationTok}[1]{\textcolor[rgb]{0.56,0.35,0.01}{\textbf{\textit{#1}}}}
\newcommand{\ErrorTok}[1]{\textcolor[rgb]{0.64,0.00,0.00}{\textbf{#1}}}
\newcommand{\ExtensionTok}[1]{#1}
\newcommand{\FloatTok}[1]{\textcolor[rgb]{0.00,0.00,0.81}{#1}}
\newcommand{\FunctionTok}[1]{\textcolor[rgb]{0.00,0.00,0.00}{#1}}
\newcommand{\ImportTok}[1]{#1}
\newcommand{\InformationTok}[1]{\textcolor[rgb]{0.56,0.35,0.01}{\textbf{\textit{#1}}}}
\newcommand{\KeywordTok}[1]{\textcolor[rgb]{0.13,0.29,0.53}{\textbf{#1}}}
\newcommand{\NormalTok}[1]{#1}
\newcommand{\OperatorTok}[1]{\textcolor[rgb]{0.81,0.36,0.00}{\textbf{#1}}}
\newcommand{\OtherTok}[1]{\textcolor[rgb]{0.56,0.35,0.01}{#1}}
\newcommand{\PreprocessorTok}[1]{\textcolor[rgb]{0.56,0.35,0.01}{\textit{#1}}}
\newcommand{\RegionMarkerTok}[1]{#1}
\newcommand{\SpecialCharTok}[1]{\textcolor[rgb]{0.00,0.00,0.00}{#1}}
\newcommand{\SpecialStringTok}[1]{\textcolor[rgb]{0.31,0.60,0.02}{#1}}
\newcommand{\StringTok}[1]{\textcolor[rgb]{0.31,0.60,0.02}{#1}}
\newcommand{\VariableTok}[1]{\textcolor[rgb]{0.00,0.00,0.00}{#1}}
\newcommand{\VerbatimStringTok}[1]{\textcolor[rgb]{0.31,0.60,0.02}{#1}}
\newcommand{\WarningTok}[1]{\textcolor[rgb]{0.56,0.35,0.01}{\textbf{\textit{#1}}}}
\usepackage{graphicx,grffile}
\makeatletter
\def\maxwidth{\ifdim\Gin@nat@width>\linewidth\linewidth\else\Gin@nat@width\fi}
\def\maxheight{\ifdim\Gin@nat@height>\textheight\textheight\else\Gin@nat@height\fi}
\makeatother
% Scale images if necessary, so that they will not overflow the page
% margins by default, and it is still possible to overwrite the defaults
% using explicit options in \includegraphics[width, height, ...]{}
\setkeys{Gin}{width=\maxwidth,height=\maxheight,keepaspectratio}
% Set default figure placement to htbp
\makeatletter
\def\fps@figure{htbp}
\makeatother
\setlength{\emergencystretch}{3em} % prevent overfull lines
\providecommand{\tightlist}{%
  \setlength{\itemsep}{0pt}\setlength{\parskip}{0pt}}
\setcounter{secnumdepth}{-\maxdimen} % remove section numbering

\title{Preliminary\_analysis}
\author{}
\date{\vspace{-2.5em}}

\begin{document}
\maketitle

\hypertarget{r-markdown}{%
\subsection{R Markdown}\label{r-markdown}}

To get Some Sense of Data and at the same build some groundwork for Data
Collection

\begin{Shaded}
\begin{Highlighting}[]
\KeywordTok{library}\NormalTok{(dplyr)}
\end{Highlighting}
\end{Shaded}

\begin{verbatim}
## 
## Attaching package: 'dplyr'
\end{verbatim}

\begin{verbatim}
## The following objects are masked from 'package:stats':
## 
##     filter, lag
\end{verbatim}

\begin{verbatim}
## The following objects are masked from 'package:base':
## 
##     intersect, setdiff, setequal, union
\end{verbatim}

\begin{Shaded}
\begin{Highlighting}[]
\KeywordTok{library}\NormalTok{(plyr)}
\end{Highlighting}
\end{Shaded}

\begin{verbatim}
## ------------------------------------------------------------------------------
\end{verbatim}

\begin{verbatim}
## You have loaded plyr after dplyr - this is likely to cause problems.
## If you need functions from both plyr and dplyr, please load plyr first, then dplyr:
## library(plyr); library(dplyr)
\end{verbatim}

\begin{verbatim}
## ------------------------------------------------------------------------------
\end{verbatim}

\begin{verbatim}
## 
## Attaching package: 'plyr'
\end{verbatim}

\begin{verbatim}
## The following objects are masked from 'package:dplyr':
## 
##     arrange, count, desc, failwith, id, mutate, rename, summarise,
##     summarize
\end{verbatim}

\begin{Shaded}
\begin{Highlighting}[]
\KeywordTok{library}\NormalTok{(janitor)}
\end{Highlighting}
\end{Shaded}

\begin{verbatim}
## 
## Attaching package: 'janitor'
\end{verbatim}

\begin{verbatim}
## The following objects are masked from 'package:stats':
## 
##     chisq.test, fisher.test
\end{verbatim}

\begin{Shaded}
\begin{Highlighting}[]
\NormalTok{Covid_Nursing_Home <-}\StringTok{ }\KeywordTok{read.csv}\NormalTok{(}\StringTok{"~/Health_IT/Covid_Nursing_Home.csv"}\NormalTok{, }\DataTypeTok{header=}\OtherTok{FALSE}\NormalTok{)}
\NormalTok{Covid_Nursing_Home <-}\StringTok{ }\NormalTok{Covid_Nursing_Home }\OperatorTok
\StringTok{  }\KeywordTok{row_to_names}\NormalTok{(}\DataTypeTok{row_number =} \DecValTok{1}\NormalTok{)}
\NormalTok{Covid_Nursing_Home <-}\StringTok{ }\KeywordTok{rename}\NormalTok{(Covid_Nursing_Home, }\KeywordTok{c}\NormalTok{(}\StringTok{"Provider State"}\NormalTok{=}\StringTok{"State"}\NormalTok{, }\StringTok{"Residents Total COVID-19 Deaths"}\NormalTok{=}\StringTok{"total_deaths"}\NormalTok{))}

\NormalTok{Covid_Nursing_Home}\OperatorTok{$}\NormalTok{total_deaths <-}\StringTok{ }\KeywordTok{as.numeric}\NormalTok{(Covid_Nursing_Home}\OperatorTok{$}\NormalTok{total_deaths)}
\NormalTok{Covid_Nursing_Home <-}\StringTok{ }\KeywordTok{na.omit}\NormalTok{(Covid_Nursing_Home)}

\CommentTok{#mean number of deaths in nursing homes due to covid}
\KeywordTok{mean}\NormalTok{(Covid_Nursing_Home}\OperatorTok{$}\NormalTok{total_deaths)}
\end{Highlighting}
\end{Shaded}

\begin{verbatim}
## [1] 2.548424
\end{verbatim}

\hypertarget{mean-deaths-in-florida-and-distribution}{%
\subsection{Mean Deaths in florida and
distribution}\label{mean-deaths-in-florida-and-distribution}}

You can also embed plots, for example:

\begin{Shaded}
\begin{Highlighting}[]
\NormalTok{Fl <-}\StringTok{ }\NormalTok{Covid_Nursing_Home }\OperatorTok\StringTok{ }\KeywordTok{filter}\NormalTok{(State}\OperatorTok{==}\StringTok{"FL"}\NormalTok{)}
\KeywordTok{library}\NormalTok{(ggplot2)}
\NormalTok{Fl}\OperatorTok{$}\NormalTok{total_deaths <-}\StringTok{ }\KeywordTok{as.numeric}\NormalTok{(Fl}\OperatorTok{$}\NormalTok{total_deaths)}
\KeywordTok{ggplot}\NormalTok{(Fl, }\KeywordTok{aes}\NormalTok{(}\DataTypeTok{x =}\NormalTok{ State, }\DataTypeTok{y =}\NormalTok{ total_deaths)) }\OperatorTok{+}
\StringTok{  }\KeywordTok{geom_boxplot}\NormalTok{()}
\end{Highlighting}
\end{Shaded}

\includegraphics{html_preliminary_files/figure-latex/unnamed-chunk-2-1.pdf}

\begin{Shaded}
\begin{Highlighting}[]
\NormalTok{Fl <-}\StringTok{ }\KeywordTok{na.omit}\NormalTok{(Fl)}
\KeywordTok{mean}\NormalTok{(Fl}\OperatorTok{$}\NormalTok{total_deaths)}
\end{Highlighting}
\end{Shaded}

\begin{verbatim}
## [1] 1.77492
\end{verbatim}

\hypertarget{mean-number-of-deaths-at-nursing-homes-in-florida-1.77}{%
\subsection{Mean Number of Deaths at Nursing Homes in FLorida :
1.77}\label{mean-number-of-deaths-at-nursing-homes-in-florida-1.77}}

Box plot suggests to investigate nursing homes with higher deaths as
clearly distribution is heavily left skewed

As a benchmark let's investigate more than 20 deaths

\begin{Shaded}
\begin{Highlighting}[]
\NormalTok{death_}\DecValTok{20}\NormalTok{ <-}\StringTok{ }\NormalTok{Covid_Nursing_Home }\OperatorTok\StringTok{ }\KeywordTok{filter}\NormalTok{(total_deaths}\OperatorTok{>}\DecValTok{20}\NormalTok{)}
\CommentTok{# number of rows where death is more than 20 : 5566 from the original 195,000 rows}
\CommentTok{#1/40th of the nursing homes have death greater than 20}

\KeywordTok{ggplot}\NormalTok{(death_}\DecValTok{20}\NormalTok{, }\KeywordTok{aes}\NormalTok{(}\DataTypeTok{x =}\NormalTok{ State, }\DataTypeTok{y =}\NormalTok{ total_deaths)) }\OperatorTok{+}
\StringTok{  }\KeywordTok{geom_boxplot}\NormalTok{()}
\end{Highlighting}
\end{Shaded}

\includegraphics{html_preliminary_files/figure-latex/unnamed-chunk-3-1.pdf}

\hypertarget{it-can-be-inferred-from-the-box-plot-that-states-like-florida-south-carolina-and-minnesota-needs-to-be-properly-investigated-for-it-capabilities-and-also-rest-of-them}{%
\subsection{It can be inferred from the box plot that states like
Florida, South Carolina and Minnesota needs to be properly investigated
for IT capabilities and also rest of
them}\label{it-can-be-inferred-from-the-box-plot-that-states-like-florida-south-carolina-and-minnesota-needs-to-be-properly-investigated-for-it-capabilities-and-also-rest-of-them}}

\hypertarget{anova}{%
\section{ANOVA}\label{anova}}

\begin{Shaded}
\begin{Highlighting}[]
\KeywordTok{library}\NormalTok{(statsr)}
\NormalTok{res.aov <-}\StringTok{ }\KeywordTok{aov}\NormalTok{(total_deaths }\OperatorTok{~}\StringTok{ }\NormalTok{State, }\DataTypeTok{data =}\NormalTok{ death_}\DecValTok{20}\NormalTok{)}
\CommentTok{# Summary of the analysis}
\KeywordTok{summary}\NormalTok{(res.aov)}
\end{Highlighting}
\end{Shaded}

\begin{verbatim}
##               Df  Sum Sq Mean Sq F value Pr(>F)    
## State         37  130406    3524   17.48 <2e-16 ***
## Residuals   5528 1114714     202                   
## ---
## Signif. codes:  0 '***' 0.001 '**' 0.01 '*' 0.05 '.' 0.1 ' ' 1
\end{verbatim}

Assuming deaths are in normal distribution for each state, ANova
suggests that at least two of the states have varying means from one
another.

\hypertarget{main-conclusions}{%
\subsubsection{Main Conclusions}\label{main-conclusions}}

Distribution is heavily left skewed meaning some if the nursing homes
have more death than rest of them - will check IT there

Mean when more than 20 deaths is : 34

States like South Carolina, Florida, Minnesota are on investigation
radar

\end{document}
